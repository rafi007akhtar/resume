%!TEX TS-program = xelatex
\documentclass[]{friggeri-cv}
\usepackage{afterpage}
\usepackage{hyperref}
\usepackage{color}
\usepackage{xcolor}
\usepackage{tabto}

\hypersetup{
    pdftitle={Resume},
    pdfauthor={},
    pdfsubject={},
    pdfkeywords={},
    colorlinks=false,       % no lik border color
   allbordercolors=white    % white border color for all
}
\addbibresource{bibliography.bib}
\RequirePackage{xcolor}
\definecolor{pblue}{HTML}{0395DE}

\begin{document}
\header {} {Md Rafi Akhtar}
      {Systems Engineer, TCS Pune}
      
% Fake text to add separator      
\fcolorbox{white}{gray}{\parbox{\dimexpr\textwidth-2\fboxsep-2\fboxrule}{%
.....
}}

% In the aside, each new line forces a line break
\begin{aside}
  \section{Address}
    185 / 1, Park Street, Kolkata-700017
    West Bengal, India
    ~
  \section{Mobile}
    +91 9674 639 341
    ~
  \section{Mail}
    \href{mailto:md.rakhtar@tcs.com}{\textbf{md.rakhtar@tcs..com}}
    \href{mailto:alimdrafi@gmail.com}{\textbf{alimdrafi@gmail.com}}
    ~
  \section{Web Links}
    \textbf{GitHub} \href{https://github.com/rafi007akhtar}{github.com/rafi007akhtar}
    \textbf{LinkedIn} \href{https://www.linkedin.com/in/md-rafi-akhtar}{linkedin.com/in/md-rafi-akhtar}
    \textbf{Twitter} \href{https://twitter.com/rafi007akhtar}{\\twitter.com/rafi007akhtar}
    \textbf{Medium}
    \href{https://medium.com/@rafi007akhtar}{medium.com/\\@rafi007akhtar}
    \textbf{Certifications}
    \href{https://drive.google.com/open?id=1yXkhjvAwEwfuUrkH_Ks33mNrBGWDfhWL}{Drive Link}
    ~
  \section{Programming Languages}
    \textbf{JavaScript}\includegraphics[scale=0.40]{4stars.png}
    \textbf{Python}\includegraphics[scale=0.40]{4stars.png}
    \textbf{Java}\includegraphics[scale=0.40]{3stars.png}
    ~
\end{aside}

\section{Experience}
\begin{entrylist}
    \entry
    {June - July, \\ 2019}
    {Learning Enabler}
    {Tata Consultancy Services, Pune}
    {Assisted in the training and management of freshers during their ILP period, as a part of the Talent Development team. \\Technologies for training included Java, SQL/PLSQL, Unix and UI.}
    \entry
    {Feb - Apr, \\ 2018}
    {Summer Intern}
    {AscentSpark, Kolkata, India}
    {Automated Text Writeups Composition: Composing One-Liners From on Public Data on Social-Media Activities of Users. \\Technology stack included UI and Django.}
\end{entrylist}

\section{Education}
\begin{entrylist}
  \entry
    {2015 - 2019}
    {B. Tech in Computer Science \& Engineering\\}
    {University of Engineering \& Management, Kolkata}
    {Primary subjects: \\
        • Data Structures and Algorithms\\
        • Objected Oriented Programming\\
    }
  \entry
    {2014}
    {High School - Class XII}
    {St. Sebastian's School, Kolkata, India}
    {Primary subjects: Mathematics, Physics, Computer Science, Chemistry.}
   \entry
    {2012}
    {Secondary School - Class X}
    {St. Sebastian's School, Kolkata, India}
    {Primary subjects: Mathematics, Physics, Computer Applications, Chemistry.}
\end{entrylist}

\section{Projects}
\begin{entrylist}
  \entry
    {07/2019}
    {\href{https://github.com/rafi007akhtar/fend-project-memory-game}
    {Memory Game}}
    {Udacity}
    {Web-game developed using pure JavaScript with DOM that uses a deck of cards for matching and unmatching. When all cards are matched, the game is won. \href{https://rafi007akhtar.github.io/fend-project-memory-game/}{\underline{[Link]}}}
  \entry
    {04/2019}
    {\href{https://github.com/rafi007akhtar/uemcrp}{UEMCRP}}{UEM, Kolkata}
    {College Resource Planning website written in Django. Implemented as a final year project. \href{https://github.com/rafi007akhtar/uemcrp}{\underline{[Link]}}}
  \entry
    {07/2017}
    {\href{https://github.com/rafi007akhtar/Wiki-Viewer}{Wiki Viewer}}
    {Free Code Camp}
    {Wikipedia Search Engine that uses the mediawiki-api to fetch the top 10 results of a search query on Wikipedia, and render them on the page. \href{https://rafi007akhtar.github.io/Wiki-Viewer/}{\underline{[Link]}}}
  \entry
    {10/2017}
    {Banking System}
    {UEM, Kolkata}
    {Online Banking System web-app written in Django. \href{https://github.com/rafi007akhtar/CentralBank}{\underline{[Link]}}}
  \entry
    {07/2017}
    {\href{https://github.com/rafi007akhtar/ecstasia}{Ecstasia}}
    {UEM, Kolkata}
    {Website for UEM, Kolkata's very first cultural fest, Ecstasia. \href{https://rafi007akhtar.github.io/ecstasia/}{\underline{[Link]}}}
  \entry
    {08/2016}
    {\href{https://codepen.io/rafi007akhtar/full/xrXEJE}{Mini-Weather-Wizard}}
    {Free Code Camp}{Weather forecasting website that uses Dark API and Google Maps API to display details about the user's current location and weather conditions. \href{https://codepen.io/rafi007akhtar/full/xrXEJE}{\underline{[Link]}}}
    
\end{entrylist}

\section{Achievements}
\begin{itemize}
    \item Winner of website designing contest in eXabyte, 2016 (the annual tech-fest of St Xavier's College).
    \item Winner of website designing contest in Paridihi, 2017 (the annual tech-fest of Meghnad Saha Instituate of Technology).
    \item Second runner-up of story-writing contest in Kshitij, 2017 (the annual tech-fest of IIT-Kharagpur).
\end{itemize}

\section{Areas of Interest}
\begin{itemize}
    \item Web Development
    \item Data Visualization
    \item Artificial Intelligence, Machine Learning, Deep Learning
    \item Deep Reinforcement Learning
\end{itemize}

\begin{aside}
\section{Tools}
\textbf{\\Front-End Technology} {\\HTML, CSS, JavaScript, ECMA Script, jQuery, Bootstrap}
\textbf{\\Back-End Technology} {Django}
\textbf{\\Database Technology} {SQL, ORM}
\textbf{\\Version Control} {Git}
\textbf{\\Code Collaboration Platform} {\\GitHub, GitLab}
\textbf{\\Text Editors} {Visual Studio Code, Brackets}
    ~
  \section{Languages}
    \textbf{English}\includegraphics[scale=0.40]{4stars.png}
    \textbf{Hindi}\includegraphics[scale=0.40]{3stars.png}
    ~
\end{aside}
\section{Certifications}
\begin{entrylist}
  \entry
    {02/2018}
    {Neural Networks and Deep Learning}
    {deeplearning.ai, Coursera}
    {\emph{Building an Image Classifier}}
  \entry
    {06/2018}
    {Data Science: Visualization}
    {Harvard University, edX}
    {\emph{Using ggplot2 in R to visualize datasets and draw inferences}}
  \entry
    {03/2017}
    {Programming, Data Structures and Algorithms in Python\\}
    {IIT-Madras, NPTEL}
    
\end{entrylist}

\textbf{Verification.} I, Md Rafi Akhtar, do hereby proclaim that all information provided is correct and to the best of my knowledge.
\end{document}